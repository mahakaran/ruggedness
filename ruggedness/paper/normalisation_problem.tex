\documentclass[10pt, reqno]{amsart}
\setlength{\parindent}{0pt}
\usepackage{amscd}
\usepackage{verbatim,ifthen}
%\usepackage{libertine}
\usepackage{scalerel,amssymb}




\usepackage{color}
\usepackage{latexsym}
\usepackage{tikz}
\usepackage{tikz-cd}
\usepackage{mathrsfs}
\usepackage{wrapfig}
\usetikzlibrary{shapes}
\usepackage{color}
\usetikzlibrary{arrows.meta}
\usepackage{bbm}
\usetikzlibrary{matrix}
\usetikzlibrary{calc}
\usetikzlibrary{arrows,intersections}
\usepackage{pgfplots}
\usepackage{multicol}
\usepackage{array}
\newcolumntype{M}[1]{>{\centering\arraybackslash}m{#1}}
\def\heading#1{\centerline{\bf #1 }\vskip 11pt}
\newcommand\nextpage{\vfill\eject}
\usepackage[colorlinks, linkcolor=blue, citecolor=magenta, linktocpage]{hyperref}
\addtolength{\textwidth}{80pt}
\addtolength{\hoffset}{-40pt}
\renewcommand{\baselinestretch}{1.2}

\addtocontents{toc}{\setcounter{tocdepth}{1}}


\theoremstyle{plain}
\newtheorem{thm}{Theorem}% reset theorem numbering for each chapter
\newtheorem{lemma}{Lemma}
\theoremstyle{definition}
\newtheorem{defn}{Definition} % definition numbers are dependent on theorem numbers
\newtheorem{exmp}[thm]{Example}


\let\oldtocsection=\tocsection
 
\let\oldtocsubsection=\tocsubsection
 
\let\oldtocsubsubsection=\tocsubsubsection
 
\renewcommand{\tocsection}[2]{\hspace{0em}\oldtocsection{#1}{#2}}
\renewcommand{\tocsubsection}[2]{\hspace{1em}\oldtocsubsection{#1}{#2}}
\renewcommand{\tocsubsubsection}[2]{\hspace{2em}\oldtocsubsubsection{#1}{#2}}





\def\msquare{\mathord{\scalebox{0.35}[0.35]{\scalerel*{\Box}{\strut}}}}

\def\Xint#1{\mathchoice
{\XXint\displaystyle\textstyle{#1}}%
{\XXint\textstyle\scriptstyle{#1}}%
{\XXint\scriptstyle\scriptscriptstyle{#1}}%
{\XXint\scriptscriptstyle\scriptscriptstyle{#1}}%
\!\int}
\def\XXint#1#2#3{{\setbox0=\hbox{$#1{#2#3}{\int}$ }
\vcenter{\hbox{$#2#3$ }}\kern-.6\wd0}}
\def\ddashint{\Xint=}
\def\dashint{\Xint-}
\usepackage{eucal}
\usepackage{calc}  
\usepackage{enumitem} 
\usepackage{tensor}
\usepackage[]{algorithm2e}

\usepackage{graphicx,wrapfig,lipsum}
\usepackage{etoolbox}
\usepackage{marginnote}
\usepackage{lipsum}
\makeatletter
\patchcmd{\@mn@margintest}{\@tempswafalse}{\@tempswatrue}{}{}
\patchcmd{\@mn@margintest}{\@tempswafalse}{\@tempswatrue}{}{}
\reversemarginpar 
\makeatother
\usepackage{scrextend}

\makeatletter
\DeclareRobustCommand\widecheck[1]{{\mathpalette\@widecheck{#1}}}
\def\@widecheck#1#2{%
    \setbox\z@\hbox{\m@th$#1#2$}%
    \setbox\tw@\hbox{\m@th$#1%
       \widehat{%
          \vrule\@width\z@\@height\ht\z@
          \vrule\@height\z@\@width\wd\z@}$}%
    \dp\tw@-\ht\z@
    \@tempdima\ht\z@ \advance\@tempdima2\ht\tw@ \divide\@tempdima\thr@@
    \setbox\tw@\hbox{%
       \raise\@tempdima\hbox{\scalebox{1}[-1]{\lower\@tempdima\box
\tw@}}}%
    {\ooalign{\box\tw@ \cr \box\z@}}}
\makeatother



\title{Maximum Number of Local Maxima on Hamming Graph}
\author{Mahakaran Sandhu -- RSC, ANU}

\begin{document}

\maketitle


Let us begin with a few key definitions. For a graph $G$, we denote by $V(G)$ the vertex set, and by $q$ the cardinality of the vertex set. \\

\textbf{Definition 1.1.} (Complete graph). A graph $G$ is said to be complete if each pair of vertices in $V(G)$ are connected by exactly one edge. \\

\textbf{Definition 1.2.} (Distance metric). Define the distance $d$ between two vertices $d(v_i, v_j), v_i, v_j \in V(G)$ as the smallest number of edges between $v_i$ and $v_j$. \\

\textbf{Definition 1.3.} (Local maximum.) Let $G$ be a graph and let $F: V(G) \to \mathbb{R}$ be some function. A vertex $v_0 \in V(G)$ is said to be a local maximum for $f$ if, 

$$f(v_0) > f(v) \text{ for all } v \in \mathbb{M}_1$$

where $\mathbb{M}_1 := \{v \in V(G): d(v_0, v)=1\}$. \\

\textbf{Proposition 1.4.} (Number of local maxima in a complete graph).\textit{Let $G$ be a complete graph. The number of local maxima for a function $f: V(G) \to \mathbb{R}$ is at most one. }

\begin{proof} Immediate from Definition 1.1. -- 1.3. We provide a proof by contradiction below for the interested reader as follows. Suppose $G$ has 2 maxima, $v_0, v_1$. From Definition 1.1 and 1.2, $d(v_i,v_j) = 1 \; \text{ for all } v_i, v_j \in V(K)$; this implies $d(v_0, v_1)=1$. This is a contradiction of Definition 1.3; a complete graph cannot have more than one local maximum. 
\end{proof}



\textbf{Definition 1.5.} (Cartesian product of graphs). The Cartesian product of graphs $G$ and $H$, $G \msquare H$, is the Cartesian product of the vertex sets of $G$ and $H$: 
$$G \msquare H = (V(G) \times V(H), E (G \times H)) $$

where two vertices $(g,h)$ and $(g',h')$ are adjacent if and only if either: 
\begin{enumerate}
\item $g = g'$ and $h$ is adjacent to $h'$ in $H$; or
\item $h=h'$ and $g$ is adjacent to $g'$ in G.
\end{enumerate}

\textbf{Definition 1.6.} (Hamming graph). The Hamming graph $H(L, q)$ is the $L$-fold Cartesian product of the complete graph $G$ with vertex set cardinality $q$.\\

\clearpage



\textbf{Proposition 1.7.} (Number of edges in a Cartesian product of graphs). \textit{Let $G$ and $H$ be graphs, and $G \msquare H$ their Cartesian product. Then, }

$$|E(G \msquare H)| = |E(G)|\cdot |V(H)| + |E(H)|\cdot |V(G)|$$

\begin{proof}

The cardinality of a set $S$ that is the Cartesian product of two sets $A, B$ is the product of the cardinalities of $A$ and $B$, $|S| = |A|\cdot|B|$. It follows that $|V(G \msquare H)| = |V(G)|\cdot |V(H)|$. Insert good proof here. 
\end{proof}


\textbf{Proposition 1.8.} (Number of local maxima in a Cartesian product of complete graphs.) \textit{Let $G$ and $H$ be complete graphs, and $G \msquare H$ their Cartesian product. The number of local maxima for a function $f: V(G \msquare H) \to \mathbb{R}$ is at most $|V(G)|$ if $|V(G)| < |V(H)|$ or $|V(H)|$ if $|V(G)| > |V(H)|$.}

\begin{proof}
By Fiending. 
\end{proof}




\textbf{Theorem A.} \textit{Let H(L,q) be a Hamming graph. The number of local maxima for a function $f:V(H) \to \mathbb{R}$ is at most ---.}

\begin{proof}
By fiending.
\end{proof}


\clearpage














\subsection*{Definition 1. (Complete graph)} A graph $K(q)$ is complete if each pair of vertices $v,u \in V(K)$ are connected by an edge. Here, $q$ is the cardinality of the vertex set, i.e. $q = |V(K)|$. We note that a complete graph has ${q \choose 2}$ edges (proof not provided). 

\subsection*{Definition 2. (Cartesian product of graphs)} The Cartesian product of graphs $G$ and $H$, $G \msquare H$, is the Cartesian product of the vertex sets of $G$ and $H$: 
$$G \msquare H = (V(G) \times V(H), E (G \times H)) $$

where 2 vertices $(g,h)$ and $(g',h')$ are adjacent if and only if either: 
\begin{enumerate}
\item $g = g'$ and $h$ is adjacent to $h'$ in $H$; or
\item $h=h'$ and $g$ is adjacent to $g'$ in G.
\end{enumerate}



\subsection*{Definition 3. (Hamming Graph)} The Hamming graph $H(L, q)$ is the $L$-fold Cartesian product of the complete graph $K(q)$. 

\subsection*{Definition 4. (Graph function)} A graph function on a graph $G(V,E)$ is a function $f: V \rightarrow \mathbb{R}$ from the vertices to the real numbers.  


\subsection*{Definition 5. (Maximum vertex)} A vertex $V_0$ is a maximum if and only if $V_0$ has the maximum value of $f$ among its 1 unit distance neighbours $\mathcal{V}_1$, where $\mathcal{V}_1 = \{d(V_0, V_i)=1 \; \forall V_i \in V(G)\}$. 

\subsection*{Proposition 1.} \textit{The complete graph $K(q)$ can have a maximum of 1 maximum node.}

\begin{proof} (By induction). By Definition 1, $d(v,u) = 1 \; \forall v, u \in V(K)$. In conjunction with Definition 5, it follows by induction that since all vertices in $K(q)$ have distance 1, there can only be a maximum of 1 maximum. \\

(By contradiction.) Suppose $K(q)$ has 2 maxima, $v_0, v_1$. By Definition 1, $d(v,u) = 1 \; \forall v, u \in V(K)$; this implies $d(v_0, v_1)=1$. This is a contradiction of Definition 5. 
\end{proof}




\end{document}